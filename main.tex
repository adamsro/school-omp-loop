% Robert Adams CS 475

\documentclass[letterpaper,10pt]{article} %twocolumn titlepage 
\usepackage{graphicx}
\usepackage{amssymb}
\usepackage{amsmath}
\usepackage{amsthm}

\usepackage{alltt}
\usepackage{float}
\usepackage{color}
\usepackage{url}

\usepackage{balance}
\usepackage[TABBOTCAP, tight]{subfigure}
\usepackage{enumitem}
\usepackage{pstricks, pst-node}


\usepackage{geometry}
\geometry{margin=1in, textheight=8.5in} %textwidth=6in

%random comment

\newcommand{\cred}[1]{{\color{red}#1}}
\newcommand{\cblue}[1]{{\color{blue}#1}}

\usepackage{hyperref}

\def\name{Robert M Adams}

%% The following metadata will show up in the PDF properties
\hypersetup{
    colorlinks = true,
    urlcolor = black,
    pdfauthor = {\name},
    pdfkeywords = {cs745},
    pdftitle = {CS 475 Project 3: SIMD Parallelization, Autocorrelation},
    pdfsubject = {CS 475 Project 3},
    pdfpagemode = UseNone,
}


\begin{document}
\title{CS 475 Project 3: SIMD Parallelization, Autocorrelation } 
\author{Robert Adams}
\maketitle


The purpose of this experiment is to test and analyze the speed
gains from using Streaming SIMD Extensions (SSE).


\section{Results}


As seen on the graphs, on os-class.engr.oregonsstate.edu I am able to 
acheive about a 60\verb|%| speed up by using SIMD instructions to muliply
and sum large arrays of floats. The amount of work done, MFLOPS, remains
consistent for all array sizes of significance for both SIMD and non-SIMD operations. 


	Compared to my previous results for multiplication of n elements, this new dataset
  doesn't seem to be suffering from any performance drop. I haven't been able to
  achieve 700 MFLOPS as I was able to before, which may be because I have already 
  passed the performance threshold being that the algorithm is now running in
  n\^2 time. Shifting of the array adds an additional 2n to the runtime, which
  shouldnt be of any significance for significantly large values of n.

  
  I’m a little confused as to why the performance isnt closer to 4x. 
  I also ran this experiment on a rackspace cloud server which was faster but 
  still gave me only about a 60\verb|%| speed up. My only hypothesis is that much of 
  the time recorded may be spent on operations other than the actual
  multiplication.
  
  
  \subsection{System Specifications}

os-class.engr.oregonstate.edu \$ lscpu:

Architecture:          x86\_64
CPU op-mode(s):        32-bit, 64-bit
Byte Order:            Little Endian
CPU(s):                4
On-line CPU(s) list:   0-3
Thread(s) per core:    1
Core(s) per socket:    1
CPU socket(s):         4
NUMA node(s):          1
Vendor ID:             GenuineIntel
CPU family:            6
Model:                 23
Stepping:              10
CPU MHz:               2992.500
BogoMIPS:              5985.00
L1d cache:             32K
L1i cache:             32K
L2 cache:              6144K
NUMA node0 CPU(s):     0-3


\begin{figure} [ht]
    \centering
    \input{plot.tex}
\caption{Speed of SIMD Multiplications (n\^2) on an array of random floats.}
    \label{runtimes}
\end{figure}

\begin{figure} [ht]
    \centering
    \input{plot_difference.tex}
    \caption{A run with with percent speedup calculated. simd mflops/non-simd mflops} 
    \label{runtimes}
\end{figure}



\end{document}
